

\nocite{Pifsterer, U. (2018) Big Bang Art History, International Journal for Digital Art History, no. 3 (juillet)}

\nocite{Blum, H. (2010). The prospect of Oceanic Studies. PMLA/Publications of the Modern Language Association of America, 125(3), 670-677. https://doi.org/10.1632/pmla.2010.125.3.670 
}

\nocite{Casati, R., Kulvicki, J., & Zeimbekis, J. (2020). Borgesian maps. Analytic Philosophy, 63(2), 90-98. https://doi.org/10.1111/phib.12204 }

\nocite{Casati, R., Kulvicki, J., & Zeimbekis, J. (2020). Borgesian maps. Analytic Philosophy, 63(2), 90-98. https://doi.org/10.1111/phib.12204}

\nocite{Casati, R., Kulvicki, J., & Zeimbekis, J. (2020). Borgesian maps. Analytic Philosophy, 63(2), 90-98. https://doi.org/10.1111/phib.12204 }

\nocite{Clarke, D. J. (2010). Water and art: A Cross Cultural Study of water as subject and medium in modern and contemporary artistic practice. Reaktion Books}

\nocite{Doyle, A. C. (1992). The Maracot deep. Tokyo Sogensha.}

\nocite{Dupont, S. (2017). I am the ocean - arts and sciences to move from ocean literacy to passion for the ocean. Journal of the Marine Biological Association of the United Kingdom, 97(6), 1211-1213. https://doi.org/10.1017/s0025315417000376 }

\nocite{Fluence, T. E. (2014). A vocabulary of water: How water in contemporary art materialises the conditions of contemporaneity. University of Melbourne. }

\nocite{Glissant, É., Nathanaël, & Malena, A. (2018). Poetic intention. Nightboat Books. }

\nocite{Ramírez-Valdivia, M. T. (2017). Memory, Heritage, and Art Production: The Jesús Ramos Frías Art Documentation Center and the Information System on Art Practice in San Luis Potosí. Art Documentation: Journal of the Art Libraries Society of North America, 36(2), 211-221.}


\nocite{Sheehan, T., Smith, M. A., Gant, C., & Jordanous, A. (2019). Meta-Curating: Online Exhibitions Questioning Curatorial Practices in the Postdigital Age. Leonardo, 52(1), 41-47. doi: 10.1162/LEON_a_01254}

\nocite{Wasielewski, A., & Dahlgren, A. (2019). Mining Art History: Bulk Converting Nonstandard PDFs to Text to Determine the Frequency of Citations and Key Terms in Humanities Articles. The Journal of Open Humanities Data, 5(1), 6. doi: 10.5334/johd.23
}

\nocite{Fraiberger, S. P., Sinatra, R., Resch, M., Riedl, C., & Barabási, A. L. (2018). Quantifying reputation and success in art. Science, 362(6416), 825-829. doi: 10.1126/science.aau7224
}

\nocite{Crockett, D. (2019). IVPY: Iconographic Visualization inside Computational Notebooks. Code4Lib Journal, (44). Retrieved from https://journal.code4lib.org/articles/14646}

\nocite{Mercuriali, G. (2018). Digital Art History and the Computational Imagination. International Journal for Digital Art History, 2, 1-19. Retrieved from https://journals.ub.uni-heidelberg.de/index.php/dah/article/download/52660/46328
}

\nocite{Bentkowska-Kafel, A. (2016). Debating Digital Art History. Visual Resources, 32(1-2), 13-39. doi: 10.1080/01973762.2016.1141458}

\nocite{Steinberg, P. E. (2013). Of other seas: Metaphors and materialities in maritime regions. Atlantic Studies, 10(2), 156-169. https://doi.org/10.1080/14788810.2013.785192 }

\nocite{Peters, K., & Steinberg, P. (2019). The ocean in excess: Towards a more-than-wet ontology. Dialogues in Human Geography, 9(3), 293-307. https://doi.org/10.1177/2043820619872886 
}

\nocite{Osborne, P. (2013). Anywhere or not at all: Philosophy of contemporary art. Verso. }
\nocite{Morton, T. (2021). Hyperobjects philosophy and ecology after the end of the world. University of Minnesota Press. }
\nocite{Morton, T. (2018). Dark ecology: For a logic of future coexistence. Columbia University Press. }
\nocite{London, J., & Oblonskai͡a R. E. (2020). Martin Iden. Rosmėn. }
\nocite{Lem, S., Jasienko, J.-M., & Lem, S. (2021). Solaris. Actes Sud. }
\nocite{Katz, E., Light, A., & Rothenberg, D. (2000). Beneath the surface: Critical essays in the philosophy of Deep Ecology. The MIT Press. }
\nocite{Doulkaridou, E. (2018). Reframing Art History. Mit Press.}
\nocite{Terras, M., Nyhan, J., & Vanhoutte, E. (Eds.). (2018). The Routledge Companion to Digital Humanities and Art History. Routledge.}
\nocite{Bönisch, D., Loos, L., & de Wilde, M. (2018). The Curator's Machine: Clustering of Museum Collection Data through Annotation of Hidden Connection Patterns between Artworks. In Balancing between Trade and Ideology (pp. 311-320). Springer.
}
\nocite{Wasielewski, A. (2019). The Growing Pains of Digital Art History: Issues for the Study of Art Using Computational Methods. In The Shape of Data in Digital Humanities (pp. 275-298). Routledge.
}
\nocite{Velthuis, O., & Baia Curioni, S. (2018). The history of art markets: methodological considerations from art history and cultural economics. Journal of Cultural Economics, 42(2), 161-178.}
\nocite{Drucker, J. (2013). The shape of data in the digital humanities. In Debates in the Digital Humanities (pp. 229-246). University of Minnesota Press.}
\nocite{Bal, M., &amp; Bryce, N. (2001). Looking in: The art of viewing. Routledge.
}
\nocite{Bauman, Z. (2017). Liquid times: Living in an age of uncertainty. Polity Press.
}
\nocite{Blum, H. (2010). The prospect of Oceanic Studies. PMLA/Publications of the Modern Language Association of America, 125(3), 670–677. doi: 10.1632/pmla.2010.125.3.670
}
\nocite{Boero, N., & Mason, K. (Eds.). (2021). The Oxford Handbook of the sociology of body and embodiment. Oxford University Press.}
\nocite{Burdick, A., Drucker, J., Lunenfeld, P., Presner, T. S., & Schnapp, J. T. (2012). Digital_Humanities. The MIT Press.}
\nocite{Burrough, P. A., & Frank, A. U. (1996). Geographic objects with indeterminate boundaries. Taylor & Francis.}
\nocite{Buskirk, M. (2005). The contingent object of Contemporary Art. MIT.
}
\nocite{Clarke, D. (2010). Water and art: A cross-cultural study of water as subject and medium in modern and contemporary artistic practice. Reaktion Books.}
\nocite{Cusack, D. R. (2016). Art and identity at the water's edge. Routledge.
}
\nocite{Dupont, S. (2017). I am the ocean – arts and sciences to move from ocean literacy to passion for the ocean. Journal of the Marine Biological Association of the United Kingdom, 97(6), 1211–1213. doi: 10.1017/s0025315417000376}
\nocite{Fluence, T. E. (2014). A vocabulary of water: How water in contemporary art materialises the conditions of contemporaneity. University of Melbourne.
}
\nocite{Gardiner, E., & Musto, R. G. (2015). The Digital Humanities: A Primer for Students and Scholars. Cambridge University Press.}
\nocite{Helmreich, S., Roosth, S., & Friedner, M. (2016). Sounding the limits of life: Essays in the anthropology of biology and beyond. Princeton University Press.}
\nocite{Holtaway, J. (2021). World-forming and Contemporary Art. Routledge.}
\nocite{Isto, R. (2015). Organic (UN)ground in the time of biopower and hyperobjects: Conceptualizing global posthumanism in the art of Xu Bing and Gu Wenda. Journal of Contemporary Chinese Art, 2(2), 195–215. doi: 10.1386/jcca.2.2-3.195_1}
\nocite{Johnson, P. (2020). Art that lives and breathes: Conserving creatures in contemporary art. Journal of the American Institute for Conservation, 60(2-3), 175–185. doi: 10.1080/01971360.2020.1790093}
\nocite{Latour, B., & Porter, C. (1998). We have never been modern. Harvard University Press.
}
\nocite{Lestel, D. (2018). Dissolving Nature in Culture. The Philosophical Ethology of Dominique Lestel. doi: 10.4324/9781315158631-8}
\nocite{Magnason, A. (2022). On Time and Water. Open Letter Press.}
\nocite{Manning, C. D., & Schutze, H. (1999). Foundations of Statistical Natural Language Processing. MIT.
}
\nocite{Manovich, L. (2001). The Language of New Media. MIT Press.}
\nocite{Manovich, L. (2020). Cultural Analytics. The MIT Press.}

\nocite{Meillassoux, Q., Meillassoux, C. (2011). Time without Becoming. Mimesis Edizioni.}
\nocite{Meillassoux, Q., Brassier, R., & Badiou, A. (2008). After Finitude: An Essay on the Necessity of Contingency. Bloomsbury Academic.}
\nocite{Meyer, G. (2020). Ocean Inspired Mathematical Art. Journal of Mathematics and the Arts, 14(1-2), 108–110. doi: 10.1080/17513472.2020.1751576}

\nocite{Morton, T. (2007). Ecology without Nature: Rethinking Environmental Aesthetics. Harvard University Press.
}
\nocite{Morton, T. (2013). Hyperobjects: Philosophy and Ecology after the End of the World. University of Minnesota Press.
}
\nocite{Moser, K. A., & Sukla, A. C. (2020). Imagination and Art: Explorations in Contemporary Theory. Brill Rodopi.}
\nocite{Neimanis, A. (2017). Bodies of Water: Posthuman Feminist Phenomenology. Bloomsbury.}
\nocite{Neset, A. (2009). Arcadian Waters and Wanton Seas: The Iconology of Waterscapes in Nineteenth-Century Transatlantic Culture. Lang.}
\nocite{Osborne, P. (2013). Anywhere or not at all: Philosophy of contemporary art. Verso.}
\nocite{Oyarzun, V. E. (2020). Avenging nature: The role of nature in modern and contemporary art and literature. Lexington Books, an imprint of The Rowman &amp; Littlefield Publishing Group, Inc.}
\nocite{Roberts, C. (2009). The unnatural history of the sea. Island Press.}

\nocite{Ross, C. (2014). The past is the present, it's the future too: The temporal turn in contemporary art. Bloomsbury.}

\nocite{Smith, T., Enwezor, O., &amp; Condee, N. (2008). Antinomies of art and culture: Modernity, postmodernity, contemporaneity. Duke University Press.}
\nocite{Steinberg, P. E. (2001). The social construction of the Ocean. Cambridge University Press.}
\nocite{Steinberg, P. E. (2013). Of other seas: Metaphors and materialities in maritime regions. Atlantic Studies, 10(2), 156–169. doi: 10.1080/14788810.2013.785192}
\nocite{Vignemont, F. de. (2020). Mind the body: An exploration of bodily self-awareness. Oxford University Press.}
\nocite{Wen, X., &amp; White, P. (2020). The role of landscape art in cultural and national identity: Chinese and European comparisons. Sustainability, 12(13), 5472. doi: 10.3390/su12135472}
\nocite{Zhai, C. X., &amp; Massung, S. (2016). Text Data Management and Analysis: A Practical Introduction to Information Retrieval and Text Mining. Association for Computing Machinery.}